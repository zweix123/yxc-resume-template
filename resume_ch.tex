%%%%%%%%%%%%%%%%%%%%%%%%%%%%%%%%%%%%%%%%%
% Medium Length Professional CV
% LaTeX Template
% Version 2.0 (8/5/13)
%
% This template has been downloaded from:
% http://www.LaTeXTemplates.com
%
% Original author:
% Trey Hunner (http://www.treyhunner.com/)
%
% Important note:
% This template requires the resume.cls file to be in the same directory as the
% .tex file. The resume.cls file provides the resume style used for structuring the
% document.
%
%%%%%%%%%%%%%%%%%%%%%%%%%%%%%%%%%%%%%%%%%

%----------------------------------------------------------------------------------------
%	PACKAGES AND OTHER DOCUMENT CONFIGURATIONS
%----------------------------------------------------------------------------------------

\documentclass{resume} % Use the custom resume.cls style
\usepackage[left=0.75in,top=0.6in,right=0.75in,bottom=0.6in]{geometry} % Document margins
\usepackage{xeCJK}  
\usepackage{color}
\usepackage[citecolor=green
            ]{hyperref}

\name{Xxxx Xxxx} % Your name
\address{+(86) 132XXXXX975 \\ xxxxxxxx@gmail.com} % Your phone number and email
\address{北京市海淀区颐和园路5号北京大学,邮编100871}  % Your address
\begin{document}

%----------------------------------------------------------------------------------------
%	EDUCATION SECTION
%----------------------------------------------------------------------------------------

\begin{rSection}{教育}
{\textbf{北京大学}} \hfill {\em 2016.9 - 至今} \\ 
硕士研究生,网络所,GPA:xxxx(100)

{\textbf{北京大学}} \hfill {\em 2012.9 - 2016.6} \\
本科生,计算机系
\begin{rSubsection}{}{}{}{}
\item 总GPA: xxx/4.00 \ \ 专业课GPA: xxxx/4.00 
\end{rSubsection}

{\textbf{爱丁堡大学}} \hfill {\em 2015.7 - 2015.8} \\ 
信息学院交流

\end{rSection}

%----------------------------------------------------------------------------------------
%	HONORS / AWARDS SECTION
%----------------------------------------------------------------------------------------

\begin{rSection}{荣誉 / 奖项}
\begin{tabular}{ @{} >{\bfseries}l @{\hspace{6ex}} l }
NOI – 金牌 & {\em 2011.7} \\
北京大学游戏AI编程对抗赛第一名 & {\em 2015.6} \\
SIGMOD2017编程竞赛第二名 & {\em 2017.3} \\
五四青年奖学金 & {\em 2014} \\
\end{tabular}
\end{rSection}

%----------------------------------------------------------------------------------------
%	WORKING EXPERIENCE SECTION
%----------------------------------------------------------------------------------------

\begin{rSection}{工作经历}

\begin{rSubsection}{谷歌中国}{\em 2018.6 - 2018.9}
{实习生,Web Development Service \\}
{对网站内容进行分类,判断是否适用于markup标签,并给出markup推荐配置}
\item[]
\begin{itemize}
\setlength\itemsep{-0.5em}
\item[-] 编写分布式数据处理Pipeline
\item[-] 用TensorFlow训练模型,并根据结果对模型进行调整和改进
\end{itemize}
\end{rSubsection}

\begin{rSubsection}{微软互联网工程院}{\em 2017.9 - 2018.2}
{实习生,小冰组 \\}
{主要负责尝试各种NLP特征,以及实现一些工程性的pipeline,编程语言主要是C\#.}
\item[]
\begin{itemize}
\setlength\itemsep{-0.5em}
\item[-] 在C\#中手写CNN模型预测。一共写了两个版本,内存优化版本比原有版本快40倍;调用Intel-MKL C++库的版本比原有版本快2倍。
\item[-] 在自动生成评论的模型中,发现了一种效果很好的特征。
\item[-] 实现了从数据库到ElasticSearch实时对接的Pipeline,具有良好的容错性和Log机制。
\end{itemize}
\end{rSubsection}

%------------------------------------------------

\begin{rSubsection}{今日头条}{\em 2017.6 - 2017.8}
{实习生,AI-Lab组 \\}
{主要负责维护服务器端的视频消重服务,应用了Thrift-Framework框架,编程语言C++。}
\item[]
\begin{itemize}
\setlength\itemsep{-0.5em}
\item[-] 尝试了不同的视频消重特征,主要有深度特征、orb特征以及k-means聚类方法。
\item[-] 编写了一个Web界面,方便人工标数据。
\item[-] 负责一些清洗数据的工作.
\end{itemize}
\end{rSubsection}

%------------------------------------------------

\begin{rSubsection}{旷视科技(Face++)}{\em 2016.3 - 2016.9}
{实习生, 文本识别研究组 \\}
{负责维护使用模型识别文字的Pipeline,以及编写Demo。Pipeline以及Demo的服务器引擎主要由C++编写。}
\item[]
\begin{itemize}
\setlength\itemsep{-0.5em}
\item[-] 用C++实现了Opencv中的minAreaRect函数,主要算法是凸包以及旋转卡壳。由于考虑到代码要在移动端运行,因此为了轻便,不能直接集成Opencv。
\item[-]用最小费用流实现了文字成行算法,效果很好。
\item[-] 负责调整Pipeline,为不同模型编写Web端的Demo。
\end{itemize}
\end{rSubsection}

\end{rSection}

%----------------------------------------------------------------------------------------
%	PROJECTS / RESEARCH EXPERIENCE SECTION
%----------------------------------------------------------------------------------------

\begin{rSection}{项目 / 研究经历}

\begin{rSubsection}{自动调参系统} {\em 2017.12 - 2018.1}
{用Java实现了一个基于贝叶斯优化的自动调参系统,总共约2000行。\\}
{Github: \rm \url{https://github.com/xxxxxxxxx}}
\item[]
\begin{itemize}
\setlength\itemsep{-0.5em}
\item[-] 模型: Gaussian Process.
\item[-] Maximizer: Random Sampling.
\item[-] Acquisition Function: EI, LCB, GP-UCB
\end{itemize}
\end{rSubsection}

\begin{rSubsection}{SIGMOD 2017 编程竞赛}{\em 2017.1 – 2017.3}{版面排名:第1名. 线下最终测试第2名 \\}
{给定短语的字典,和很多长文档,然后要求动态地对字典进行插入、删除操作。目标是在每篇文档中,查找出现过哪些字典中的短语。总数据量~10G。提交代码,根据代码运行时间排名}
\item[]
\begin{itemize}
\setlength\itemsep{-0.5em}
\item[-] 使用Hash表和Trie建索引。(尝试过AC-自动机,效果不好)
\item[-] 实现了一套复杂的并行机制,使得插入、删除、查询操作可以做到完全并行。
\item[-] 编程语言C++,总共约3000行。
\end{itemize}
\end{rSubsection}

\begin{rSubsection}{阿里大数据竞赛} {\em 2016.10 - 2016.12}
{给定2016.1-2016.6的线下优惠券使用情况,要求预测2016.7线下优惠券的使用情况,初赛排名(1/1500). 复赛排名(19/1500) \\}
{三人组队, 负责模型和特征选择,以及数据清洗。}
\item[]
\begin{itemize}
\setlength\itemsep{-0.5em}
\item[-] 复赛时学习使用阿里数加平台。主要工作分为两部分:用Sql处理数据,以及用Java编写MapReduce处理数据;
\item[-] 尝试过LR、随机森林、Xgboost等机器学习模型,以及模型融合;
\item[-] 尝试加入时序特征,使AUC提高了1\%左右;
\end{itemize}
\end{rSubsection}

\begin{rSubsection}{3D联机对战手游}{\em 2016.7 – 2016.8}{独立开发 \\}
{一个MOBA类的安卓手游,每局游戏支持4人在线对战,提供了玩家积分机制,以及根据战斗力自动匹配的机制。}
\item[]
\begin{itemize}
\setlength\itemsep{-0.5em}
\item[-] 游戏逻辑部分由Unity3D引擎开发,编程语言是 C\#;
\item[-] 服务器端选择了阿里云Linux服务器,登录系统和积分系统由Django框架开发;
\item[-] 总代码量约7000行;
\end{itemize}
\end{rSubsection}

\begin{rSubsection}{Java代码和注释的自动匹配}{\em 2015.7 – 2015.8}{暑期在爱丁堡大学交换了2个月。 \\}
{在Charles Sutton的指导下,使用深度学习匹配Java代码块和注释块。最终目标是Java代码注释自动补全。}
\item[]
\begin{itemize}
\setlength\itemsep{-0.5em}
\item[-] 数据集是从Github上选的Java工程;
\item[-] 编写Java语法分析器,提取Java代码块和与之对应的注释块。
\item[-] 生成训练数据:尝试各种规则,生成匹配好的<代码,注释>对。
\end{itemize}
\end{rSubsection}

\end{rSection}

%----------------------------------------------------------------------------------------
%	SKILLS SECTION
%----------------------------------------------------------------------------------------

\begin{rSection}{技能}
\begin{rSubsection}
{}{}{}{}
\item[-] 编程语言: 熟练使用 C/C++, C\#, java, python和Go语言, 了解shell, matlab和一些函数式语言,比如scala和scheme。
\item[-] 处理大数据: 熟悉Hadoop、Spark,以及Parameter Server。
\item[-] 机器学习:熟练使用numpy, matplotlib, sklearn和xgboost, 了解pandas和tensorflow。
\end{rSubsection}
\end{rSection}

%----------------------------------------------------------------------------------------
%	OTHERS SECTION
%----------------------------------------------------------------------------------------

\begin{rSection}{其他}
\begin{rSubsection}
{}{}{}{}
\item[-] 数学水平: 数学分析、高等代数和离散数学等均A或A+
\item[-] 英语水平: CET-6.
\item[-] GitHub: https://github.com/xxxxxx
\end{rSubsection}
\end{rSection}

\end{document}
